\section{Issues with Hadoop}

While we can acknowledge Hadoop is a fantastic tool and has paved the way for big data processing, its not without it's issues. One research paper outlined one such issue\cite[pg.3]{hadoop-issues}. The issue is related to how social media and how they process huge amounts of loosely connected data. 

Take the example of Google+ friend circles, where a user is presented with a list of possible friend connections to add to their circle. This is computed in some MapReduce process that checks for friends of friends that you may have something in common with. The problem is that if any piece of data changes, such as a friend un-friending one of your friends, then the suggestions may not be accurate at a particular time until the MapReduce process has run again. Because the process has to run again it uses up more a lot of resources. 

One potential solution presented in the paper was to use a form of caching\cite[pg.3-4]{hadoop-issues}. For example, if we know some friends of friends are inactive for a certain amount of time we could potentially not include them in the MapReduce process. If enough people are inactive there is a potential to speed up the computational process.